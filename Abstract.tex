\begin{center}
Abstract \\
\textsc{Energy Correlators as a probe of the hard process in Run 24 $p-p$ collisions at $\sqrt{s}=200$ $GeV$ with the sPHENIX detector at RHIC} \\
by \\
\textsc{Skadi K Grossberndt} \\[0.25in]
\end{center}

\vspace{0.25in}

\noindent Adviser: Professor Stefan Bathe

\vspace{0.25in}

\noindent This dissertation consists of three parts in addition to a literature review establishing the state of the art of jet physics and the Energy-Energy correlator at high energies\ldots

\paragraph{Part 1: Hardware} %\input{One.tex}
This part discusses the physical hardware used to make the measurements of the energy in the jets created by the proton-proton collisions. 
This part contains a discussion of the sPHENIX detector and an in-depth look at the relevant susbsystems for the meausements in this analysis. 
In addition, the Monte Carlo methods and models to discern the expected respnse and calibrate the detector are discussed in detail, with both subparts coming together in a disucssion of backgrounds and error calculation in general and for the observables at hand.
\paragraph{Part 2: Technicalities} %\input{Two.tex}
This part builds from first principles up to the jet measurement in a theoretical light, and then discusses the additional computational techniques on display that will provide the bridge between the theory of these observables and the practical application to sPHENIX. 
This part continains a chapter that is a pared down discussion of a midstream paper proposed as part of the thesis process that establishes the safety of this observable against jet finding choices.
\paragraph{Part 3: Experimental Output} %\input{Three.tex}
This part is the meet and potatoes of the dissertation, discussing results from the experiment and the analysis in light of the previous parts and comparing to the world results from related systems.

