\documentclass[letterpaper, 12pt, oneside]{book}
\usepackage[margin = 1in, includehead, footskip=0.25in]{geometry}
\usepackage{setspace}
\doublespacing
\usepackage{amsfonts}
\usepackage{amsmath}
\usepackage{amsthm}
\usepackage{tikz}
\usepackage{graphicx}
\usepackage{multirow}
\usepackage{booktabs}
\usepackage{tabularx}
\usepackage{longtable}
\usepackage{lscape}
\usepackage{import}
\usepackage[style=phys]{biblatex}
\bibliography{thesis_refs}
%\addbibresource{thesis_refs.bib}
\usepackage{lineno}
\linenumbers
\usepackage[pdfauthor={My Full Name},
    pdftitle={Title},
    hidelinks
]{hyperref}
\newtheorem{lemma}{Lemma}[section]
\theoremstyle{definition}
\newtheorem{definition}{Definition}[section]
\newtheorem{corolary}{Corolary}[section]
%\author{Skadi K Grossberndt}
%\title{Energy Correlators as a probe of the hard process in Run 24 $p-p$ collisions at $\sqrt{s}=200$ $GeV$ with the sPHENIX detector at RHIC}
%\affil{Graduate Center and Baruch College, CUNY}
\begin{document}
%\maketitle
\frontmatter
\begin{titlepage}

\begin{center}

~\vspace{2in}

\textsc{Energy Correlators as a probe of the hard process in Run 24 $p-p$ collisions at $\sqrt{s}=200$ $GeV$ with the sPHENIX detector at RHIC} \\[0.5in]
by \\[0.5in]
\textsc{Skadi Kurt Grossberndt} 

\vspace{\fill}
A dissertation submitted to the Graduate Faculty in Physics in partial fulfillment of the requirements for the degree of Doctor of Philosophy, The City University of New York \\[0.25in]
2025

\end{center}

\end{titlepage}

\setcounter{page}{2}
\phantom{}\vspace{\fill}
\begin{center}
\copyright~2025\\
\textsc{Skadi Kurt Grossberndt}\\
All Rights Reserved\\
\end{center}

\begin{center}
This manuscript has been read and accepted by the Graduate Faculty in Physics in satisfaction of the dissertation requirement for the degree of Doctor of Philosophy.
\end{center}

\vspace{0.75in}

\begin{tabular}{p{1.75in}p{0.5in}p{3.5in}}
~                                   & & \textbf{Professor Stefan Bathe}\\
~                                   & & \\
\hrulefill                          & &\hrulefill \\
Date                                & & Chair of Examining Committee\\
~                                   & & \\
~                                   & & \textbf{Professor Tom Jones}\\
~                                   & & \\
\hrulefill                          & &\hrulefill \\
Date                                & & Executive Officer\\
\end{tabular}

\vspace{0.75in}

\begin{tabular}{l}
\textbf{Professor Adrian Dimitru} \\
\textbf{Professor Raghav K} \\
\textbf{Professor Jamal Jallian-Marian} \\
\textbf{Professor Z} \\
Supervisory Committee \\
\end{tabular}


\vspace{\fill}
\begin{center}
\textsc{The City University of New York}
\end{center}

\begin{center}
Abstract \\
\textsc{Energy Correlators as a probe of the hard process in Run 24 $p-p$ collisions at $\sqrt{s}=200$ $GeV$ with the sPHENIX detector at RHIC} \\
by \\
\textsc{Skadi K Grossberndt} \\[0.25in]
\end{center}

\vspace{0.25in}

\noindent Adviser: Professor Stefan Bathe

\vspace{0.25in}

\noindent This dissertation consists of three parts in addition to a literature review establishing the state of the art of jet physics and the Energy-Energy correlator at high energies\ldots

\paragraph{Part 1: Hardware} %\input{One.tex}
This part discusses the physical hardware used to make the measurements of the energy in the jets created by the proton-proton collisions. 
This part contains a discussion of the sPHENIX detector and an in-depth look at the relevant susbsystems for the meausements in this analysis. 
In addition, the Monte Carlo methods and models to discern the expected respnse and calibrate the detector are discussed in detail, with both subparts coming together in a disucssion of backgrounds and error calculation in general and for the observables at hand.
\paragraph{Part 2: Technicalities} %\input{Two.tex}
This part builds from first principles up to the jet measurement in a theoretical light, and then discusses the additional computational techniques on display that will provide the bridge between the theory of these observables and the practical application to sPHENIX. 
This part continains a chapter that is a pared down discussion of a midstream paper proposed as part of the thesis process that establishes the safety of this observable against jet finding choices.
\paragraph{Part 3: Experimental Output} %\input{Three.tex}
This part is the meet and potatoes of the dissertation, discussing results from the experiment and the analysis in light of the previous parts and comparing to the world results from related systems.


\include{Acknowledgements}

\tableofcontents
\listoftables
\listoffigures

\mainmatter
\chapter{Literature Review}
\label{ch:LR}
%\documenclass[Skadi_Thesis_Draft.tex]{subfiles}
%\begin{document}
\section{Jets Definitions}
The study of Quantum Chromodynamics in high energy collisions, such as those at the Relativistic Heavy Ion Collider or the Large Hadron Collider, is often carried out through investigation of the kinematics of jets.
Jets, as objects, sit in the boundary between experiment and theory, being an experimental signature corresponding to final states of quarks and gluons produced in collisions. 
Jets are a cluster of final state particles that result from the showering and hadronization of the initial parton, that are identified in experiment through use of one of the multiple jet reconstruction algorithms. 
\subsection{Jet Reconstruction Algorithms}
In Ryan Atkin's paper \textit{Review of jet reconstruction algorithms} \cite{Atkin2015}, Atkin provides an overview of the standard algorithms with a focus on practical implementation for experimental usage. %this is a very basic conference note style papaer, should probably just use it as an intro to each method
Atkin discusses a number of algorithms that fall into two categories: Cone based and Sequential Clustering 
\subsubsection{Cone Methods}
	In Atkin's paper, Atkin discusses two iterative cone procedures--Iterative Cone with Progressive Removal and with Split Merge--and one more modern cone method, Seedless Infrared Safe Cone. 
	These iterative cone procedures are described as simple to implement in an experimental context, however, they both suffer deficiencies from a theoretical perspective as they fail to be IRC safe \cite{Atkin2015} \cite{Salam2007}.

\subsection{Quark and Gluon Jets}	
	In their paper \textit{An Operational Definition of Quark and Gluon Jets} \cite{Komiske2018} Kimiske, Metodiev and Thaler present a theoretical definition of jet flavor with an eye towards application at colliders. 
	They aim to present a precise and practical definition of jet flavor at the hadron level, thus allowing for direct measurement from collider experiments rather than a measurement by proxy. 
	While the gluon or quark jet is well defined at the level of the hard scattering, corresponding to an initiating parton of the primary vertex \cite{Jones1989} \cite{Fodor1990}. 
	Experimentally, there is significant interest in being able to discriminate quark and gluon jets in searches for Beyond Standard Model physics, searches for Dark Matter, in jet substructure analysis and in QGP physics. \cite{Gallicchio2011}\cite{Lima2017}\cite{Bhattacherjee2017}\cite{}. 
	The determination of relative populations of quark and gluon jets in a dijet sample is one of the primary tasks of this thesis and will be discussed at length in chapters \ref{ch:alphas} and \ref{ch:qgdiscr}. 

	Kimiske et al.--from here referred to as KMT--set their work in the context of discerning the quark-gluon jet populations in samples rather than applying an event-by-event labeling scheme. 
	KMT's work 
\section{Jet Substructure Measurements}

\section{N-Point Energy Correlator}
%\end{document}
 
\part{Hardware: The Detector and Simulations}

\chapter{RHIC and BNL}
\chapter{The sPHENIX detector}
\chapter{Monte Carlo Simulations}
\chapter{Determining Backgrounds and Errors}
\part{Technicalities: The finer points of theory and computing}
\chapter{The Energy Correlator and the primary vertex}
\chapter{Jets in Vacuum and the PDF}
\section{Jet Identification Algorithms}
As discused in chapter \ref{ch:LR}, there are a variety of jet finding algorithims that prioritize different theoretical aspacts of the underlying physics while being experimentally realizable \cite{Dokshitzer1997} \cite{Atkin2015}. 

In general, a jet identification algorithim needs to be IRC safe. 
That is, the jet object needs to display invariance in the Infrared (IR) and Collinear regimes, managing real-virtual cancellation and keeping results meaningful for emmission and splitting respectively. 

\chapter{So exactly how intelligent is AI}
\chapter{Proof Solving and Validation as Quality Control Mechanisms}
\part{Experimental Output: The Main Event}
\chapter{So is sPHENIX actually working?}
\chapter{Measuring the Energy Correlator}
\chapter{The Power of the ENC: $\alpha_s$ at the few GeV scale}
\chapter{Event-by-Event distinguishing}
\chapter{Comparison is the Theft of Joy: What does the LHC say, and how about STAR?}
\chaptermark{LHC and STAR Results}
\chapter{Entaglement and other Lofty Goals}
\part{Wrapping it all up}
\chapter{Observable prospects for Run 25 and the EIC}
\chapter{Remaining Questions}
\chapter{Implementation and application for the remaining sPHENIX data}
\chaptermark{Future Work}
\singlespacing
\printbibliography
%\bibliographystyle{apalike}
%\bibliography{thesis}
\end{document}
