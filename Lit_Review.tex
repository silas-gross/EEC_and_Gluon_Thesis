%\documenclass[Skadi_Thesis_Draft.tex]{subfiles}
%\begin{document}
\section{Jets Definitions}
The study of Quantum Chromodynamics in high energy collisions, such as those at the Relativistic Heavy Ion Collider or the Large Hadron Collider, is often carried out through investigation of the kineatics of jets.
Jets, as objects, sit in the boundary between experiment and theory, being an experimental signature corresponding to final states of quarks and gluons produced in collisions. 
Jets are a cluster of final state particles that result from the showering and hadronization of the inital parton, that are identified in experiment through use of one of the multiple jet reconstruction algorithims. 
\subsection{Jet Reconstruction Algorithims}
In Ryan Atkin's paper \textit{Review of jet reconstruction algorithms} \cite{Atkin2015}, Atkin provides an overview of the standard algorithms with a focus on pratical implementation for experimental useage. 
\section{Jet Substucure Measurements}
\section{N-Point Energy Correlator}
%\end{document}
