%\documenclass[Skadi_Thesis_Draft.tex]{subfiles}
%\begin{document}
\section{Jets Definitions}
The study of Quantum Chromodynamics in high energy collisions, such as those at the Relativistic Heavy Ion Collider or the Large Hadron Collider, is often carried out through investigation of the kinematics of jets.
Jets, as objects, sit in the boundary between experiment and theory, being an experimental signature corresponding to final states of quarks and gluons produced in collisions. 
Jets are a cluster of final state particles that result from the showering and hadronization of the initial parton, that are identified in experiment through use of one of the multiple jet reconstruction algorithms. 
\subsection{Jet Reconstruction Algorithms}
In Ryan Atkin's paper \textit{Review of jet reconstruction algorithms} \cite{Atkin2015}, Atkin provides an overview of the standard algorithms with a focus on practical implementation for experimental usage. %this is a very basic conference note style papaer, should probably just use it as an intro to each method
Atkin discusses a number of algorithms that fall into two categories: Cone based and Sequential Clustering 
\subsubsection{Cone Methods}
	In Atkin's paper, Atkin discusses two iterative cone procedures--Iterative Cone with Progressive Removal and with Split Merge--and one more modern cone method, Seedless Infrared Safe Cone. 
	These iterative cone procedures are described as simple to implement in an experimental context, however, they both suffer deficiencies from a theoretical perspective as they fail to be IRC safe \cite{Atkin2015} \cite{Salam2007}.

\subsection{Quark and Gluon Jets}	
	In their paper \textit{An Operational Definition of Quark and Gluon Jets} \cite{Komiske2018} Kimiske, Metodiev and Thaler present a theoretical definition of jet flavor with an eye towards application at colliders. 
	They aim to present a precise and practical definition of jet flavor at the hadron level, thus allowing for direct measurement from collider experiments rather than a measurement by proxy. 
	While the gluon or quark jet is well defined at the level of the hard scattering, corresponding to an initiating parton of the primary vertex \cite{Jones1989} \cite{Fodor1990}. 
	Experimentally, there is significant interest in being able to discriminate quark and gluon jets in searches for Beyond Standard Model physics, searches for Dark Matter, in jet substructure analysis and in QGP physics. \cite{Gallicchio2011}\cite{Lima2017}\cite{Bhattacherjee2017}\cite{}. 
	The determination of relative populations of quark and gluon jets in a dijet sample is one of the primary tasks of this thesis and will be discussed at length in chapters \ref{ch:alphas} and \ref{ch:qgdiscr}. 

	Kimiske et al.--from here referred to as KMT--set their work in the context of discerning the quark-gluon jet populations in samples rather than applying an event-by-event labeling scheme. 
	KMT's work 
\section{Jet Substructure Measurements}

\section{N-Point Energy Correlator}
%\end{document}
